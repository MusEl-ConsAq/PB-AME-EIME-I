\documentclass[a4paper,12pt]{article}
\usepackage{ME_AQ_temp}
\usepackage{tabularx}
\usepackage{listings}
\usepackage{xcolor}

\setmonofont{Fira Code}
% \setmonofont{Source Code Pro}
% \setmonofont{DejaVu Sans Mono}
% \setmonofont{JuliaMono} % Ottimo per caratteri scientifici e matematici

\usepackage{listings}
\usepackage{microtype}
\usepackage[backend=biber, style=authoryear-comp, sorting=ynt]{biblatex}

\addbibresource{bibliography.bib}

% \sloppy % Prova a rimuoverlo o commentarlo. Usa emergencystretch che è meno aggressivo.
\emergencystretch=2em % Ho aumentato leggermente il valore, puoi regolarlo

\lstset{
    % Il font di base ora sarà quello impostato da \setmonofont
    basicstyle=\ttfamily\footnotesize, 
    breaklines=true,
    frame=single,
    numbers=left,
    numbersep=5pt,
    postbreak=\mbox{\textcolor{red}{$\hookrightarrow$}\space},
    breakindent=1.5em,
    breakautoindent=true,
}

%% -------------------------------------- %%
%  - impostare il titolo della tesina in entrambe le righe 
% 
%% -------------------------------------- %%

\newcommand{\mycustomtitle}{Titolo della Tesina}
% Definisci un titolo personalizzato
\newcommand{\setmytitle}[1]{\renewcommand{\mycustomtitle}{#1}}

% Definizione del titolo e dell'autore
\title{Corsi Accademici di Musica Elettronica DCSL34 Conservatorio A. Casella, L'Aquila \\ \vspace{0.5cm} \fontsize{14}{17}\bfseries\uppercase{"I am sitting in a room" di Alvin Lucier}}
\author{Pietro Barale \\ esame di \\\bfseries{Analisi, Esecuzione e Interpretazione della Musica Elettroacustica}}
\date{05/07/2025}

% Sovrascrivi le impostazioni di hyperref per l'indice
\hypersetup{
    linkcolor=black, % Imposta il colore dei link dell'indice a nero
}

\begin{document}

% Pagina 1: Titolo e riassunto
\maketitle
\thispagestyle{empty}

\begin{center}
    \vspace{1cm}
    \textbf{\fontsize{12}{15}\selectfont{Sommario}}
\end{center}

Il presente elaborato analizza la composizione elettroacustica I Am Sitting in a Room (1969) di Alvin Lucier, indagandola come opera paradigmatica che ridefinisce il concetto stesso di materiale musicale e di processo compositivo. Attraverso l'analisi della partitura verbale, del dispositivo tecnico e del suo fondamento concettuale, la tesina esplora il processo iterativo di registrazione e riproduzione che trasforma gradualmente un testo parlato nelle frequenze di risonanza naturali dello spazio in cui l'opera viene realizzata.

Il fulcro dell'analisi dimostra come il vero materiale compositivo non risieda né nella voce né nell'acustica ambientale prese singolarmente, bensì nella loro interazione dinamica e reciproca. La voce agisce da catalizzatore e lo spazio da filtro attivo, in un sistema autonomo che sposta il ruolo del compositore da artefice di forme sonore a "architetto di processi". Viene inoltre esaminata la figura dell'interprete, il quale assume una responsabilità co-creativa nel determinare i parametri esecutivi e il punto di conclusione formale dell'opera.

In conclusione, I Am Sitting in a Room emerge non come un oggetto musicale fisso, ma come un modello generativo. La sua identità si fonda sulla fedeltà al principio processuale piuttosto che sul risultato acustico, proponendo un approccio alla creazione musicale che privilegia il sistema sull'oggetto e apre nuovi spazi per la pratica contemporanea.
\newpage
% Genera l'indice
\tableofcontents  

% Pagina 2: Introduzione e resto del testo
\newpage

% --- File di inclusione generato automaticamente ---
% --- Contenuto LaTeX autogenerato da capitolo1.md (sezione 1) ---

\section{Note Bibliografiche e Contesto storico}
Alvin Lucier (1931-2021) nacque a Nashua, New Hampshire, dove completò la formazione primaria. Proseguì gli studi presso la Portsmouth Abbey School e successivamente si laureò a Yale e Brandeis University. Grazie a una prestigiosa borsa di studio Fulbright, trascorse due anni di perfezionamento a Roma.

Durante il suo soggiorno in Europa, ebbe modo di entrare in contatto con le avanguardie musicali del tempo, specialmente Milano e Darmstadt, che negli anni seguenti influenzarono profondamente il suo approccio compositivo.

La carriera accademica di Lucier iniziò nel 1962 presso la Brandeis University, dove insegnò fino al 1970 dirigendo il Brandeis University Chamber Chorus, ensemble specializzato nell'esecuzione di musica contemporanea. Durante questo periodo, nel 1966, co-fondò la Sonic Arts Union insieme a Robert Ashley, David Behrman e Gordon Mumma, collettivo che divenne fondamentale per lo sviluppo della musica sperimentale americana.

Dal 1968 al 2011 insegnò presso la Wesleyan University, dove ricoprì la cattedra John Spencer Camp di Musica, formando generazioni di giovani compositori. Parallelamente all'attività didattica, mantenne un'intensa attività concertistica e di conferenze in Asia, Europa e Stati Uniti.

Lucier è stato un pioniere in molti ambiti della composizione, concentrando gran parte della sua produzione sulle proprietà fisiche del suono stesso: la risonanza degli spazi, l'interferenza di fase tra toni con accordatura ravvicinata e la trasmissione del suono attraverso mezzi fisici. Come osservò il compositore James Tenney, Lucier apparteneva a quella rara categoria di compositori il cui lavoro \textit{è così convincente e al tempo stesso così diverso da quello dei suoi contemporanei e predecessori da costringerci a rivedere i nostri assunti fondamentali sulla musica}.

La prima opera significativa in cui esplorò le caratteristiche fenomenologiche del suono fu \textit{Music for Solo Performer}(1965), sottotitolata \textit{for enormously amplified brain waves and percussion}. Come sottolineò Lucier, questo brano segnò una svolta nel suo percorso. Durante la sua direzione del Brandeis University Chamber Chorus, incontrò il fisico Edmond Dewan, che gli propose di sperimentare con apparecchiature per elettroencefalogrammi. Lucier rimase affascinato dalle onde alfa (8-13 Hz), inudibili all'orecchio umano. Rifiutando di trasporle o registrarle su nastro, ideò un sistema che le amplificava al punto da far risuonare strumenti percussivi disposti sul palco, senza bisogno di musicisti.

L'esperienza con 'Music for Solo Performer' aprì a Lucier una nuova comprensione del suono come fenomeno fisico misurabile, studiando in che modo le onde attraversano lo spazio. Come disse in un'intervista:

\textit{Pensare ai suoni come lunghezze d'onda misurabili, anziché come note musicali, ha trasformato la mia idea di musica da metafora a fatto concreto, collegandomi all'architettura.}

Questa nuova prospettiva lo portò a esplorare sistematicamente l'acustica degli spazi. In Vespers(1968), i musicisti bendati si muovevano con dispositivi di ecolocalizzazione, producendo click direzionali i cui echi rivelavano la conformazione dell'ambiente. L'anno successivo, in I Am Sitting in a Room (1969), utilizzò un processo ancora più diretto per far emergere le frequenze di risonanza naturali di uno spazio attraverso registrazioni in serie della propria voce.  % Auto-generated: include sezione1.tex
% --- Contenuto LaTeX autogenerato da capitolo2.md (sezione 2) ---

\section{\textit{I am sitting in a room}}
L'opera fu composta e registrata per la prima volta nel 1969 presso l'Electronic Music Studio della Brandeis University. Una seconda registrazione fu realizzata nel marzo 1970, nell'appartamento di Lucier a Middletown, Connecticut. La prima esecuzione pubblica ebbe luogo sempre nel 1970, al Guggenheim Museum di New York.

L'idea nacque dopo che un collega aveva raccontato a Lucier di aver assistito a una conferenza al MIT, durante la quale l'ingegnere statunitense Amar Bose descriveva il metodo che utilizzava per testare le caratteristiche degli altoparlanti che stava sviluppando. Il metodo consisteva nel far confluire nei dispositivi l'audio da essi stessi prodotto in precedenza, registrato dai microfoni, generando così un processo iterativo.

Il brano consiste nella lettura di un breve testo, che viene registrata da un microfono. La registrazione viene poi riprodotta nella stanza attraverso un altoparlante e nuovamente registrata. Questo processo viene ripetuto più volte e, a causa delle specifiche caratteristiche acustiche della stanza in cui si esegue il brano (dimensioni, geometria, materiali), alcune frequenze risonanti vengono enfatizzate mentre altre vengono attenuate. Alla fine, le parole diventano incomprensibili e vengono sostituite dalla risonanza caratteristica dello spazio.
\subsection{Partitura}
La composizione è strutturata sotto forma di \textit{partitura verbale}, un documento dettagliato con specifiche istruzioni e linee guida per la realizzazione del lavoro. La partitura di \textit{I am sitting in a room}, pubblicata nel 1995 e identica all'originale scritto dal compositore tra il 1969 e il 1970, consiste in due pagine scritte. Sotto il titolo, l'intestazione riporta la dicitura \textit{per voce e nastro elettromagnetico} (for Voice and electromagnetic tape).

Il materiale richiesto per la realizzazione dell'opera, indicato subito dopo il titolo, comprende:

\begin{itemize}
    \item un microfono
    \item due registratori
    \item un amplificatore
    \item un altoparlante
\end{itemize}

All'esecutore inoltre viene data la possibilità di scegliere il luogo per la realizzazione in base alle qualità acustiche desiderate (choose a room the musical qualities of which you would like to evoke), nonché di utilizzare il testo proposto oppure un qualsiasi altro testo, senza limiti di lunghezza (use the following text or any other text of any length).

Il testo proposto e utilizzato nelle realizzazioni del compositore stesso, è il seguente:

I am sitting in a room different from the one you are in now. I am recording the sound of my speaking voice and I am going to play it back into the room again and again until the resonant frequencies of the room reinforce themselves so that any semblance of my speech, perhaps with the exception of rhythm, is destroyed. What you will hear, then, are the natural resonant frequencies of the room articulated by speech. I regard this activity not so much as a demonstration of a physical fact, but more as a way to smooth out any irregularities my speech might have.

In italiano può essere tradotto come:

Mi trovo in una stanza diversa da quella in cui voi vi trovate ora. Sto registrando il suono della mia voce parlata e la riascolterò nella stanza più e più volte, finché le frequenze risonanti della stanza si rinforzeranno al punto che ogni somiglianza con il mio discorso, forse ad eccezione del ritmo, verrà distrutta. Ciò che udrete, allora, sono le frequenze risonanti naturali della stanza articolate attraverso il parlato. Considero questa attività non tanto come una dimostrazione di un fatto fisico, quanto piuttosto come un modo per eliminare le eventuali irregolarità del mio parlato.

Procedimento tecnico:

\begin{itemize}
    \item Collegare il microfono all'ingresso del primo registratore (attach the microphone to the input of tape recorder \#1)
\end{itemize}

\begin{itemize}
    \item Connettere l'uscita del secondo registratore all'amplificatore e all'altoparlante (connect the output of tape recorder \#2 to the amplifier and loudspeaker)
\end{itemize}

Dopodiché vengono illustrate le modalità esecutive del brano, con precise indicazioni procedurali:

\begin{itemize}
    \item Registra la tua voce su nastro, attraverso il microfono collegato al primo registratore (record your voice on tape through the microphone attached to tape recorder \#1)
\end{itemize}

\begin{itemize}
    \item Riavvolgi il nastro, portalo al secondo registratore e riproducilo nella stanza attraverso l'altoparlante e registra una seconda generazione della traccia originale utilizzando nuovamente il microfono collegato al primo registratore. (Rewind the tale to ius beginning, transfer it to tape recorder 2\#, play it back into the room through the loudspeaker and record a second generation of the original recorder statement through the microphone attached to tape recorder \#1)
\end{itemize}

\begin{itemize}
    \item Riavvolgi questa seconda generazione all'inizio e uniscila alla fine della registrazione originale presente sul secondo registratore (Rewind the second generation to its beginning and splice it onto the end of the original recorded statement on tape recorder \#2)
\end{itemize}

\begin{itemize}
    \item Riproduci solo la seconda generazione nella stanza attraverso l'altoparlante e registra una terza generazione dell'enunciato originale attraverso il microfono collegato al primo registratore. Prosegui questo processo attraverso molte generazioni. Tutte le generazioni, unite insieme in ordine cronologico, compongono un brano su nastro la cui durata è determinata dalla lunghezza dell'enunciato originale e dal numero di generazioni registrate (play the second generation only back in the room through the loudspeaker and record a third generation of the original recorded statement through the microphone attached to tape recorder \#1. Continue this process through many generations. All the generations spliced together in chronological order make a tape composition the length of which is determined by the length of the original statement and the number of generations recorded.)
\end{itemize}

Portando a termine queste istruzioni si otterrà una realizzazione completa del brano su nastro (monofonico). L'esecuzione sarà quindi la diffusione sonora del nastro ottenuto.

Il compositore conclude la partitura contemplando diverse varianti:

\begin{itemize}
    \item Versioni che siano eseguibili in tempo reale (make version that can be performed in real time).
\end{itemize}

\begin{itemize}
    \item Versioni in cui il microfono viene spostato in diversi punti della stessa stanza o in stanze differenti (make versions in which, for each generation, the microphone is moved to different parts of the room or rooms);
\end{itemize}

\begin{itemize}
    \item Versioni con uno o più interpreti che utilizzano lingue diverse in stanze diverse (make versions using one or more speakers of different languages in different rooms);
\end{itemize}

\begin{itemize}
    \item Versioni in cui una singola registrazione del testo viene utilizzata in diversi ambienti (make versions in which one recorded statement is recycled through many rooms).
\end{itemize}

\subsection{Configurazioni tecniche e modalità esecutive}

Lucier utilizzò due registratori Revox A77, magnetofoni professionali che rappresentavano l'eccellenza della tecnologia analogica dell'epoca. La scelta di questi strumenti non fu casuale: I Revox A77 garantivano stabilità di velocità e alto rapporto segnale/rumore, permettendo di isolare il fenomeno della risonanza ambientale da altre forme di degrado, inevitabile in un processo basato su generazioni successive di registrazione analogica.

Le prime versioni dell'opera, realizzate tra il 1969 e il 1970, seguirono un approccio \textit{per fasi}. Lucier iniziava con una registrazione iniziale del testo parlato, realizzata in condizioni controllate su nastro magnetico professionale. Il processo iterativo seguiva poi una metodologia: la registrazione veniva riprodotta dal primo Revox A77 attraverso l'altoparlante, mentre il secondo registratore catturava simultaneamente il risultato su un nuovo nastro. Al termine di ogni ciclo, il compositore interrompeva il processo per sostituire il nastro sorgente con la nuova registrazione, che diventava così la base per l'iterazione successiva. Questo approccio permetteva un controllo qualitativo di ogni generazione, consentendo al compositore di monitorare l'evoluzione spettrale del materiale e di determinare il punto ottimale di conclusione del processo.

L'evoluzione verso l'esecuzione \textit{in tempo reale} rappresentò una rivoluzione concettuale dell'opera. In questa modalità, i due registratori Revox A77 operavano in una configurazione di feedback continuo: mentre un registratore riproduceva costantemente il materiale, l'altro registrava simultaneamente su nastro a loop o attraverso un sistema di registrazione continua. Questa configurazione eliminava le interruzioni tra le generazioni, creando un flusso sonoro ininterrotto dove la trasformazione del materiale avveniva in tempo reale. Il compositore manteneva il controllo solo sui parametri di amplificazione e bilanciamento, mentre l'evoluzione del processo dipendeva esclusivamente dall'interazione dinamica tra il segnale e le caratteristiche acustiche dell'ambiente.

In questa modalità la realizzazione dell'opera coincideva con la sua esecuzione dal vivo e le condizioni circostanziali assumevano un ruolo significativo in quanto contributo essenziale per la natura \textit{dal vivo}. Questa dimensione introduceva elementi di imprevedibilità legati alle specifiche condizioni ambientali di ogni esecuzione: variazioni di temperatura, umidità, presenza del pubblico e caratteristiche architettoniche dello spazio influenzavano direttamente il risultato sonoro. In una condizione controllata 'in vitro', esecuzione e realizzazione dell'opera si articolano in due fasi distinte: la messa a punto delle procedure tecniche e la successiva presentazione del risultato. Nella versione in tempo reale, invece, queste due pratiche coincidono completamente.

\subsection{Materiale Compositivo}

La definizione del \textit{materiale} in \textit{I Am Sitting in a Room} richiede un'analisi che va oltre la concezione tradizionale del materiale musicale. Come evidenziato nella partitura stessa, l'interprete si fornisce del materiale scegliendo la stanza di cui evocare le «qualità musicali» e il testo da registrare. Questa scelta rivela una bipolarità morfologica sostanziale: da una parte l'articolazione fonica complessa del parlato (spettro armonico delle vocali, rumore delle consonanti, elementi prosodici), dall'altra le risonanze naturali della sala (frequenze proporzionali alle dimensioni e forme geometriche, profilo spettrale modellato dalle proprietà di assorbimento acustico delle superfici).

Il primo elemento rappresenta un fenomeno dinamico, percepito diacronicamente come sequenza di suoni nel tempo. Il secondo costituisce un fenomeno vissuto sincronicamente, percepito come l'insieme delle risonanze che ogni spazio aggiunge istantaneamente a qualsiasi evento sonoro. Tuttavia, il vero materiale compositivo dell'opera non è né la voce iniziale né le caratteristiche acustiche dello spazio prese singolarmente, ma il processo stesso di trasformazione che li mette in relazione.

La voce parlata, funziona come catalizzatore per rivelare le proprietà dello spazio acustico. Il testo proposto da Lucier descrive esattamente il processo che si sta compiendo, eliminando qualsiasi elemento di significato poetico o narrativo esterno. Come osserva il compositore, questa attività è \textit{non tanto una dimostrazione di un fatto fisico, quanto piuttosto un modo per eliminare le eventuali irregolarità del mio parlato} (I regard this activity not so much as a demonstration of a physical fact, but more as a way to smooth out any irregularities my speech might have). La dissoluzione progressiva del contenuto linguistico nelle frequenze di risonanza rappresenta quindi il cuore del processo compositivo.

Nonostante l'uso dell'equipaggiamento tecnico (Revox A77) sia stato cruciale per Lucier, l'opera mantiene la sua validità concettuale anche con strumentazione di qualità inferiore, che non cambia la natura fondamentale del processo di rivelazione delle proprietà acustiche dello spazio. La ridefinizione del concetto di \textit{materiale musicale} rappresenta una delle rivoluzioni più profonde dell'opera. Il vero materiale compositivo non è né la voce iniziale né le caratteristiche acustiche dello spazio prese singolarmente, ma il processo stesso di trasformazione che li mette in relazione. Il materiale di \textit{I Am Sitting in a Room} è dunque costituito dall'interazione dinamica tra segnale e ambiente, un processo che trasforma ciò che abitualmente \textit{non viene mai percepito} - la risonanza ambientale - nel protagonista assoluto dell'esperienza musicale.  % Auto-generated: include sezione2.tex
% --- Contenuto LaTeX autogenerato da capitolo3.md (sezione 3) ---

\section{Feedback e trasformazione}
Il processo di trasformazione del materiale sonoro in \textit{I Am Sitting in a Room} si articola attraverso un meccanismo di feedback che coinvolge simultaneamente la voce parlata e le caratteristiche acustiche dello spazio. La voce iniziale, sottoposta al ciclo iterativo di registrazione-riproduzione, si fonde progressivamente con le risonanze naturali della stanza fino a dissolversi nelle frequenze caratteristiche dello spazio stesso.

Questa fusione avviene attraverso un processo di retroazione (feedback) che può essere formalizzato come una linea di ritardo con coefficiente di amplificazione. In termini matematici, il sistema si comporta come un filtro IIR (infinite impulse response) dove:

y(n) = A y(n-k) + x(n)

dove  y(n) rappresenta il segnale in uscita, che è pari alla somma del segnale attuale in ingresso x(n) e di quello fornito in uscita ad un dato tempo precedente y(n-k) che viene attenuato da un coefficiente A<1 (per non incorrere in accumulazione e saturazione). Si tratta di una linea di ritardo con feedback dove il ritardo è dato da k e il valore di feedback è A. Volendo riscrivere l'equazione rispetto alla realizzazione di Lucier, dove A rappresenta il valore di ingresso al secondo registratore e B il livello di uscita dal primo registratore:

y(n) = A y(n-k) + B x(n)

La durata del testo recitato assume un ruolo determinante nel processo. Come osservato da Lucier, stabilisce non solo la frequenza fondamentale teorica del processo (1/k), ma determina anche la \textit{finestra temporale} entro la quale le risonanze della stanza possono svilupparsi e decadere. Il rapporto tra coefficiente di feedback e durata del ciclo è fondamentale poiché definisce l'equilibrio tra persistenza del processo e controllo della saturazione: un valore di feedback troppo elevato causerebbe accumulazione fino alla saturazione, mentre un valore troppo basso farebbe estinguere rapidamente il processo di retroazione.

Il processo presenta caratteristiche \textit{autonome} che sollevano interrogativi fondamentali circa il controllo artistico. Una volta attivato, il sistema procede secondo le proprie leggi fisiche, ma il risultato finale dipende dalle specifiche caratteristiche della stanza, dalla qualità della voce, dalle condizioni ambientali, e da una serie di parametri che non possono essere completamente controllati o previsti. La distinzione tra processo  automatico e processo  autonomo è qui fondamentale: mentre un processo automatico segue istruzioni predeterminate senza possibilità di variazione, un processo autonomo sviluppa dinamiche proprie in interazione con l'ambiente e le condizioni di contorno.

Il processo è potremo definirlo stocastico: la lenta mutazione timbrica è sin dall'inizio predestinata a svolgersi in una sola direzione, ad uno stadio finale che in linea teorica è noto sin dall'inizio. Tuttavia, per ciascuna stanza e per ciascun testo adottati, il destino del processo non cambia ma il suo fenomeno rimane caratterizzato da variazioni che dipendono dalle variazioni nelle condizioni di contorno. Il compositore non controlla i dettagli dello svolgimento ma progetta le condizioni entro le quali il processo può svilupparsi autonomamente. L'intenzionalità si manifesta nella scelta dei parametri iniziali, nella selezione dello spazio, nella formulazione del testo iniziale. La casualità — intesa come insieme delle variazioni non controllabili che ogni esecuzione introduce — non rappresenta un elemento di disturbo ma costituisce il materiale stesso dell'opera, garantendo che ogni realizzazione sia un evento unico e irripetibile.
\subsection{Articolazione}
La tecnologia musicale sviluppata da Lucier nella sua opera, va oltre la tradizionale dicotomia mezzo/fine. I mezzi tecnici, da strumenti di riproduzione diventano strumenti di produzione e di pensiero, passando da una destinazione funzionale ad una destinazione creativa. La tecnologia non è più supporto neutro, ma diventa co-autore dell'opera insieme all'interprete e all'ambiente, generando fenomeni musicali che non sono né ottenibili né immaginabili con mezzi diversi.

L'architettura del processo si basa su un elemento fondamentale: la stanza che agisce come filtro posto tra ingresso e uscita del sistema. Delle frequenze presenti nello spettro della voce, risuoneranno in particolare quelle coincidenti con le proprie risonanze acustiche naturali. L'azione di filtro attuata dalla stanza deriva dalle interferenze costruttive (risonanze) e distruttive (antirisonanze) relative alle riflessioni sonore delle superfici presenti.

Il minimalismo di Lucier si manifesta in questa grande economia di mezzi abbinata a un pensiero estremamente chiaro circa la loro funzione compositiva. \textit{I am sitting in a room} fornisce un esempio molto fertile e caratteristico dove l'opera consiste con chiarezza nella messa in atto di un metodo di lavoro, mentre il fenomeno sonoro offerto all'ascolto consiste nella traccia udibile di questo mettere in atto.
\subsection{Voce e Spazio: reciprocità dinamica}
Nel corso del brano, la bipolarità caratteristica del materiale - la netta distinzione morfologica tra voce e stanza (suono e spazio) - viene dinamizzata attraverso il processo. Voce e spazio operano l'uno sull'altro: ciascuno dei due agisce sull'altro e lo modifica mentre viene dall'altro agito e modificato. Le risonanze della stanza vengono «articolate» attraverso la voce, che diventa così non solo materiale sonoro ma anche mezzo per sollecitare le caratteristiche acustiche dello spazio. Reciprocamente, le risonanze hanno per Lucier il compito di levigare o smussare l'articolazione della voce, trasformando la stanza in uno strumento che riduce progressivamente l'articolazione dinamica del parlato fino a produrre una \textit{texture} di frequenze lentamente variabile.

Questa interazione ridefinisce radicalmente il concetto di materialità musicale: voce e spazio, nella loro relazione dinamica, non sono affatto materiale neutro, perché è proprio lo svolgersi della loro interferenza nel tempo che si ascolta nell'intero svolgimento del brano. In quanto materiali particolari e non generici, voce e stanza sono mezzi: ciascuno dei due ha la funzione di mediare, a chi ascolta, qualcosa che riguarda l'altro. Il rapporto tra materiali sonori e processo realizzativo crea una bipolarità: l'ambiente è contemporaneamente mezzo di produzione e materiale compositivo, modificando dinamicamente voce e spazio.  % Auto-generated: include sezione3.tex
% --- Contenuto LaTeX autogenerato da capitolo4.md (sezione 4) ---

\section{Interprete e ascoltatore}
La figura dell'interprete in \textit{I am sitting in a room} non può essere facilmente categorizzata come semplice esecutore, strumento o co-compositore. L'interprete incarnava piuttosto tutte e tre queste funzioni simultaneamente. La partitura verbale di Lucier fornisce le istruzioni necessarie per una o più realizzazioni ma lascia spazio all'interpretazione musicale. Questa apertura controllata assegna all'interprete responsabilità che vanno oltre la semplice esecuzione di istruzioni.

L'interprete deve controllare una serie di parametri interconnessi: dosare i livelli di riproduzione e registrazione, trovare l'equilibrio tra durata del testo e numero di iterazioni, posizionare microfono e altoparlante. Tutti questi elementi sono collegati tra loro: la durata del brano, i tempi di evoluzione del processo, le risonanze che emergeranno dipendono dalla voce utilizzata, dal contenuto fonetico del testo, dal livello dell'altoparlante, dalla distanza del microfono. L'interprete deve soppesare queste decisioni musicalmente, considerando l'interazione tra tutti i fattori.

Nella mia realizzazione, ho utilizzato il \textit{sistema Lucier} come strumento compositivo per creare un brano acusmatico. Ho implementato il processo attraverso una patch in Pure Data, utilizzando oggetti delwrite\textasciitilde{} e delread\textasciitilde{} con un buffer di 15000 campioni, un controllo del feedback impostato a 0.7 e un monitoraggio in tempo reale. Ho condotto il processo in tre stanze diverse, ognuna con caratteristiche acustiche specifiche. In ciascuna delle tre stanze ho eseguito multiple versioni del brano variando sistematicamente la distanza tra microfono e altoparlante: 1 metro, 2 metri e 3 metri. Per ogni configurazione di distanza ho realizzato diverse esecuzioni, ottenendo risultati sonori differenti che riflettevano non solo le caratteristiche acustiche specifiche di ogni ambiente, ma anche l'influenza determinante della distanza microfono-altoparlante sui tempi di sviluppo del processo e sulle frequenze di risonanza emergenti. In ogni ambiente ho portato il processo fino alla dissoluzione completa della voce, scegliendo solo le ultime \textit{generazioni} dove ormai il parlato era completamente irriconoscibile. Il risultato finale del mio acusmatico è un montaggio con i tre esiti sonori sovrapposti, ciascuno rappresentante le frequenze di risonanza caratteristiche di ogni spazio con le specifiche configurazioni di distanza che hanno prodotto i risultati più significativi. Questa realizzazione dimostra come il sistema compositivo di Lucier possa diventare uno strumento da cui partire per nuove possibilità che mantengono identità sonore diverse rispetto al brano originario, pur mantenendo gli stessi processi.

Il problema dell'identità dell'opera emerge chiaramente quando ci si confronta con realizzazioni che utilizzano il sistema compositivo di Lucier come strumento per nuove creazioni. La realizzazione acusmatica ottenuta montando i risultati finali di tre stanze diverse - dove il processo di \textit{I am sitting in a room} è stato portato al punto di dissoluzione completa della voce in ciascun ambiente - solleva interrogativi fondamentali sui confini dell'opera. È mediante le differenze introdotte dall'interprete che si manifesta l'identità del sistema costruito dal compositore, e ciascuna realizzazione consiste in una manifestazione diversa dello stesso potenziale costruttivo. In questa prospettiva, il \textit{sistema Lucier} diventa uno strumento compositivo che mantiene la poetica originale pur generando identità sonore inedite.

Nella versione in tempo reale, l'ascoltatore assume un ruolo attivo nel processo. La sua presenza fisica influenza direttamente il risultato sonoro: variazioni di temperatura, umidità, movimenti nello spazio diventano parte del processo di trasformazione. L'interprete deve reagire a eventi imprevisti mantenendo il controllo del feedback per raggiungere l'obiettivo del brano.

Emerge inoltre nell'esecuzione del brano un'ulteriore domanda: quando termina l'opera? Il brano termina quando il suono della voce iniziale è ormai dileguato e le risonanze della stanza si sono manifestate chiaramente, e quando, nel passaggio da una \textit{generazione} alla successiva, il suono non sembra più modificarsi in alcun modo percettivamente sensibile. Questa decisione, affidata al giudizio dell'interprete sulla base di un criterio di \textit{efficienza musicale}, introduce un elemento di soggettività che contrasta con l'apparente automatismo del processo, rivelando come anche in un sistema deterministico l'elemento umano rimanga decisivo per la definizione dei confini formali dell'opera.  % Auto-generated: include sezione4.tex
% --- Contenuto LaTeX autogenerato da capitolo5.md (sezione 5) ---

\section{Conclusione}
La partitura verbale di Lucier rappresenta un modello: fornisce le informazioni necessarie per ricreare il processo mantenendo aperti gli spazi per l'interpretazione. \textit{I am sitting in a room} solleva interrogativi fondamentali sulla concezione stessa dell'opera musicale. È mediante le differenze introdotte dall'interprete che si manifesta l'identità del sistema costruito dal compositore. In questa prospettiva, non esistono \textit{versioni} di un'unica opera ma piuttosto diverse opere generate dallo stesso sistema compositivo. La mia realizzazione acusmatica, ottenuta dal montaggio di tre stanze diverse, si inserisce in questa dinamica, utilizzando il \textit{sistema Lucier} come strumento.

L'opera di Lucier dimostra come sia possibile concepire la composizione non solo come costruzione di oggetti sonori ma anche come progettazione di sistemi capaci di generare esperienze musicali uniche. In questa prospettiva, l'identità dell'opera si preserva non attraverso la ripetizione identica ma attraverso la fedeltà al principio generativo, aprendo spazi inesplorati per la creatività contemporanea dove il compositore diventa architetto di processi piuttosto che costruttore di forme.  % Auto-generated: include sezione5.tex


\nocite{LucierSimon1980}
\nocite{Davis2003}
\nocite{DiScipio2005}
\nocite{GalanteSani2000}
\nocite{Kahn2009}

\nocite{Lucier1994online}


% Aggiungi la bibliografia
\newpage % ---- Inizia una nuova pagina prima della bibliografia
\printbibliography[nottype=online, title={Bibliografia}]

% 2. Stampa la SITOGRAFIA (SOLO le voci di tipo @online)
% Il titolo della sezione sarà "Sitografia"
\printbibliography[type=online, title={Sitografia}]

\end{document}
