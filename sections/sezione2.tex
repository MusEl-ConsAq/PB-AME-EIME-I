% --- Contenuto LaTeX autogenerato da capitolo2.md (sezione 2) ---

\section{\textit{I am sitting in a room}}
L'opera fu composta e registrata per la prima volta nel 1969 presso l'Electronic Music Studio della Brandeis University. Una seconda registrazione fu realizzata nel marzo 1970, nell'appartamento di Lucier a Middletown, Connecticut. La prima esecuzione pubblica ebbe luogo sempre nel 1970, al Guggenheim Museum di New York.

L'idea nacque dopo che un collega aveva raccontato a Lucier di aver assistito a una conferenza al MIT, durante la quale l'ingegnere statunitense Amar Bose descriveva il metodo che utilizzava per testare le caratteristiche degli altoparlanti che stava sviluppando. Il metodo consisteva nel far confluire nei dispositivi l'audio da essi stessi prodotto in precedenza, registrato dai microfoni, generando così un processo iterativo.

Il brano consiste nella lettura di un breve testo, che viene registrata da un microfono. La registrazione viene poi riprodotta nella stanza attraverso un altoparlante e nuovamente registrata. Questo processo viene ripetuto più volte e, a causa delle specifiche caratteristiche acustiche della stanza in cui si esegue il brano (dimensioni, geometria, materiali), alcune frequenze risonanti vengono enfatizzate mentre altre vengono attenuate. Alla fine, le parole diventano incomprensibili e vengono sostituite dalla risonanza caratteristica dello spazio.
\subsection{Partitura}
La composizione è strutturata sotto forma di \textit{partitura verbale}, un documento dettagliato con specifiche istruzioni e linee guida per la realizzazione del lavoro. La partitura di \textit{I am sitting in a room}, pubblicata nel 1995 e identica all'originale scritto dal compositore tra il 1969 e il 1970, consiste in due pagine scritte. Sotto il titolo, l'intestazione riporta la dicitura \textit{per voce e nastro elettromagnetico} (for Voice and electromagnetic tape).

Il materiale richiesto per la realizzazione dell'opera, indicato subito dopo il titolo, comprende:

\begin{itemize}
    \item un microfono
    \item due registratori
    \item un amplificatore
    \item un altoparlante
\end{itemize}

All'esecutore inoltre viene data la possibilità di scegliere il luogo per la realizzazione in base alle qualità acustiche desiderate (choose a room the musical qualities of which you would like to evoke), nonché di utilizzare il testo proposto oppure un qualsiasi altro testo, senza limiti di lunghezza (use the following text or any other text of any length).

Il testo proposto e utilizzato nelle realizzazioni del compositore stesso, è il seguente:

I am sitting in a room different from the one you are in now. I am recording the sound of my speaking voice and I am going to play it back into the room again and again until the resonant frequencies of the room reinforce themselves so that any semblance of my speech, perhaps with the exception of rhythm, is destroyed. What you will hear, then, are the natural resonant frequencies of the room articulated by speech. I regard this activity not so much as a demonstration of a physical fact, but more as a way to smooth out any irregularities my speech might have.

In italiano può essere tradotto come:

Mi trovo in una stanza diversa da quella in cui voi vi trovate ora. Sto registrando il suono della mia voce parlata e la riascolterò nella stanza più e più volte, finché le frequenze risonanti della stanza si rinforzeranno al punto che ogni somiglianza con il mio discorso, forse ad eccezione del ritmo, verrà distrutta. Ciò che udrete, allora, sono le frequenze risonanti naturali della stanza articolate attraverso il parlato. Considero questa attività non tanto come una dimostrazione di un fatto fisico, quanto piuttosto come un modo per eliminare le eventuali irregolarità del mio parlato.

Procedimento tecnico:

\begin{itemize}
    \item Collegare il microfono all'ingresso del primo registratore (attach the microphone to the input of tape recorder \#1)
\end{itemize}

\begin{itemize}
    \item Connettere l'uscita del secondo registratore all'amplificatore e all'altoparlante (connect the output of tape recorder \#2 to the amplifier and loudspeaker)
\end{itemize}

Dopodiché vengono illustrate le modalità esecutive del brano, con precise indicazioni procedurali:

\begin{itemize}
    \item Registra la tua voce su nastro, attraverso il microfono collegato al primo registratore (record your voice on tape through the microphone attached to tape recorder \#1)
\end{itemize}

\begin{itemize}
    \item Riavvolgi il nastro, portalo al secondo registratore e riproducilo nella stanza attraverso l'altoparlante e registra una seconda generazione della traccia originale utilizzando nuovamente il microfono collegato al primo registratore. (Rewind the tale to ius beginning, transfer it to tape recorder 2\#, play it back into the room through the loudspeaker and record a second generation of the original recorder statement through the microphone attached to tape recorder \#1)
\end{itemize}

\begin{itemize}
    \item Riavvolgi questa seconda generazione all'inizio e uniscila alla fine della registrazione originale presente sul secondo registratore (Rewind the second generation to its beginning and splice it onto the end of the original recorded statement on tape recorder \#2)
\end{itemize}

\begin{itemize}
    \item Riproduci solo la seconda generazione nella stanza attraverso l'altoparlante e registra una terza generazione dell'enunciato originale attraverso il microfono collegato al primo registratore. Prosegui questo processo attraverso molte generazioni. Tutte le generazioni, unite insieme in ordine cronologico, compongono un brano su nastro la cui durata è determinata dalla lunghezza dell'enunciato originale e dal numero di generazioni registrate (play the second generation only back in the room through the loudspeaker and record a third generation of the original recorded statement through the microphone attached to tape recorder \#1. Continue this process through many generations. All the generations spliced together in chronological order make a tape composition the length of which is determined by the length of the original statement and the number of generations recorded.)
\end{itemize}

Portando a termine queste istruzioni si otterrà una realizzazione completa del brano su nastro (monofonico). L'esecuzione sarà quindi la diffusione sonora del nastro ottenuto.

Il compositore conclude la partitura contemplando diverse varianti:

\begin{itemize}
    \item Versioni che siano eseguibili in tempo reale (make version that can be performed in real time).
\end{itemize}

\begin{itemize}
    \item Versioni in cui il microfono viene spostato in diversi punti della stessa stanza o in stanze differenti (make versions in which, for each generation, the microphone is moved to different parts of the room or rooms);
\end{itemize}

\begin{itemize}
    \item Versioni con uno o più interpreti che utilizzano lingue diverse in stanze diverse (make versions using one or more speakers of different languages in different rooms);
\end{itemize}

\begin{itemize}
    \item Versioni in cui una singola registrazione del testo viene utilizzata in diversi ambienti (make versions in which one recorded statement is recycled through many rooms).
\end{itemize}

\subsection{Configurazioni tecniche e modalità esecutive}

Lucier utilizzò due registratori Revox A77, magnetofoni professionali che rappresentavano l'eccellenza della tecnologia analogica dell'epoca. La scelta di questi strumenti non fu casuale: I Revox A77 garantivano stabilità di velocità e alto rapporto segnale/rumore, permettendo di isolare il fenomeno della risonanza ambientale da altre forme di degrado, inevitabile in un processo basato su generazioni successive di registrazione analogica.

Le prime versioni dell'opera, realizzate tra il 1969 e il 1970, seguirono un approccio \textit{per fasi}. Lucier iniziava con una registrazione iniziale del testo parlato, realizzata in condizioni controllate su nastro magnetico professionale. Il processo iterativo seguiva poi una metodologia: la registrazione veniva riprodotta dal primo Revox A77 attraverso l'altoparlante, mentre il secondo registratore catturava simultaneamente il risultato su un nuovo nastro. Al termine di ogni ciclo, il compositore interrompeva il processo per sostituire il nastro sorgente con la nuova registrazione, che diventava così la base per l'iterazione successiva. Questo approccio permetteva un controllo qualitativo di ogni generazione, consentendo al compositore di monitorare l'evoluzione spettrale del materiale e di determinare il punto ottimale di conclusione del processo.

L'evoluzione verso l'esecuzione \textit{in tempo reale} rappresentò una rivoluzione concettuale dell'opera. In questa modalità, i due registratori Revox A77 operavano in una configurazione di feedback continuo: mentre un registratore riproduceva costantemente il materiale, l'altro registrava simultaneamente su nastro a loop o attraverso un sistema di registrazione continua. Questa configurazione eliminava le interruzioni tra le generazioni, creando un flusso sonoro ininterrotto dove la trasformazione del materiale avveniva in tempo reale. Il compositore manteneva il controllo solo sui parametri di amplificazione e bilanciamento, mentre l'evoluzione del processo dipendeva esclusivamente dall'interazione dinamica tra il segnale e le caratteristiche acustiche dell'ambiente.

In questa modalità la realizzazione dell'opera coincideva con la sua esecuzione dal vivo e le condizioni circostanziali assumevano un ruolo significativo in quanto contributo essenziale per la natura \textit{dal vivo}. Questa dimensione introduceva elementi di imprevedibilità legati alle specifiche condizioni ambientali di ogni esecuzione: variazioni di temperatura, umidità, presenza del pubblico e caratteristiche architettoniche dello spazio influenzavano direttamente il risultato sonoro. In una condizione controllata 'in vitro', esecuzione e realizzazione dell'opera si articolano in due fasi distinte: la messa a punto delle procedure tecniche e la successiva presentazione del risultato. Nella versione in tempo reale, invece, queste due pratiche coincidono completamente.

\subsection{Materiale Compositivo}

La definizione del \textit{materiale} in \textit{I Am Sitting in a Room} richiede un'analisi che va oltre la concezione tradizionale del materiale musicale. Come evidenziato nella partitura stessa, l'interprete si fornisce del materiale scegliendo la stanza di cui evocare le «qualità musicali» e il testo da registrare. Questa scelta rivela una bipolarità morfologica sostanziale: da una parte l'articolazione fonica complessa del parlato (spettro armonico delle vocali, rumore delle consonanti, elementi prosodici), dall'altra le risonanze naturali della sala (frequenze proporzionali alle dimensioni e forme geometriche, profilo spettrale modellato dalle proprietà di assorbimento acustico delle superfici).

Il primo elemento rappresenta un fenomeno dinamico, percepito diacronicamente come sequenza di suoni nel tempo. Il secondo costituisce un fenomeno vissuto sincronicamente, percepito come l'insieme delle risonanze che ogni spazio aggiunge istantaneamente a qualsiasi evento sonoro. Tuttavia, il vero materiale compositivo dell'opera non è né la voce iniziale né le caratteristiche acustiche dello spazio prese singolarmente, ma il processo stesso di trasformazione che li mette in relazione.

La voce parlata, funziona come catalizzatore per rivelare le proprietà dello spazio acustico. Il testo proposto da Lucier descrive esattamente il processo che si sta compiendo, eliminando qualsiasi elemento di significato poetico o narrativo esterno. Come osserva il compositore, questa attività è \textit{non tanto una dimostrazione di un fatto fisico, quanto piuttosto un modo per eliminare le eventuali irregolarità del mio parlato} (I regard this activity not so much as a demonstration of a physical fact, but more as a way to smooth out any irregularities my speech might have). La dissoluzione progressiva del contenuto linguistico nelle frequenze di risonanza rappresenta quindi il cuore del processo compositivo.

Nonostante l'uso dell'equipaggiamento tecnico (Revox A77) sia stato cruciale per Lucier, l'opera mantiene la sua validità concettuale anche con strumentazione di qualità inferiore, che non cambia la natura fondamentale del processo di rivelazione delle proprietà acustiche dello spazio. La ridefinizione del concetto di \textit{materiale musicale} rappresenta una delle rivoluzioni più profonde dell'opera. Il vero materiale compositivo non è né la voce iniziale né le caratteristiche acustiche dello spazio prese singolarmente, ma il processo stesso di trasformazione che li mette in relazione. Il materiale di \textit{I Am Sitting in a Room} è dunque costituito dall'interazione dinamica tra segnale e ambiente, un processo che trasforma ciò che abitualmente \textit{non viene mai percepito} - la risonanza ambientale - nel protagonista assoluto dell'esperienza musicale.