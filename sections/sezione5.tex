% --- Contenuto LaTeX autogenerato da capitolo5.md (sezione 5) ---

\section{Conclusione}
La partitura verbale di Lucier rappresenta un modello: fornisce le informazioni necessarie per ricreare il processo mantenendo aperti gli spazi per l'interpretazione. \textit{I am sitting in a room} solleva interrogativi fondamentali sulla concezione stessa dell'opera musicale. È mediante le differenze introdotte dall'interprete che si manifesta l'identità del sistema costruito dal compositore. In questa prospettiva, non esistono \textit{versioni} di un'unica opera ma piuttosto diverse opere generate dallo stesso sistema compositivo. La mia realizzazione acusmatica, ottenuta dal montaggio di tre stanze diverse, si inserisce in questa dinamica, utilizzando il \textit{sistema Lucier} come strumento.

L'opera di Lucier dimostra come sia possibile concepire la composizione non solo come costruzione di oggetti sonori ma anche come progettazione di sistemi capaci di generare esperienze musicali uniche. In questa prospettiva, l'identità dell'opera si preserva non attraverso la ripetizione identica ma attraverso la fedeltà al principio generativo, aprendo spazi inesplorati per la creatività contemporanea dove il compositore diventa architetto di processi piuttosto che costruttore di forme.