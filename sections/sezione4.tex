% --- Contenuto LaTeX autogenerato da capitolo4.md (sezione 4) ---

\section{Interprete e ascoltatore}
La figura dell'interprete in \textit{I am sitting in a room} non può essere facilmente categorizzata come semplice esecutore, strumento o co-compositore. L'interprete incarnava piuttosto tutte e tre queste funzioni simultaneamente. La partitura verbale di Lucier fornisce le istruzioni necessarie per una o più realizzazioni ma lascia spazio all'interpretazione musicale. Questa apertura controllata assegna all'interprete responsabilità che vanno oltre la semplice esecuzione di istruzioni.

L'interprete deve controllare una serie di parametri interconnessi: dosare i livelli di riproduzione e registrazione, trovare l'equilibrio tra durata del testo e numero di iterazioni, posizionare microfono e altoparlante. Tutti questi elementi sono collegati tra loro: la durata del brano, i tempi di evoluzione del processo, le risonanze che emergeranno dipendono dalla voce utilizzata, dal contenuto fonetico del testo, dal livello dell'altoparlante, dalla distanza del microfono. L'interprete deve soppesare queste decisioni musicalmente, considerando l'interazione tra tutti i fattori.

Nella mia realizzazione, ho utilizzato il \textit{sistema Lucier} come strumento compositivo per creare un brano acusmatico. Ho implementato il processo attraverso una patch in Pure Data, utilizzando oggetti delwrite\textasciitilde{} e delread\textasciitilde{} con un buffer di 15000 campioni, un controllo del feedback impostato a 0.7 e un monitoraggio in tempo reale. Ho condotto il processo in tre stanze diverse, ognuna con caratteristiche acustiche specifiche. In ciascuna delle tre stanze ho eseguito multiple versioni del brano variando sistematicamente la distanza tra microfono e altoparlante: 1 metro, 2 metri e 3 metri. Per ogni configurazione di distanza ho realizzato diverse esecuzioni, ottenendo risultati sonori differenti che riflettevano non solo le caratteristiche acustiche specifiche di ogni ambiente, ma anche l'influenza determinante della distanza microfono-altoparlante sui tempi di sviluppo del processo e sulle frequenze di risonanza emergenti. In ogni ambiente ho portato il processo fino alla dissoluzione completa della voce, scegliendo solo le ultime \textit{generazioni} dove ormai il parlato era completamente irriconoscibile. Il risultato finale del mio acusmatico è un montaggio con i tre esiti sonori sovrapposti, ciascuno rappresentante le frequenze di risonanza caratteristiche di ogni spazio con le specifiche configurazioni di distanza che hanno prodotto i risultati più significativi. Questa realizzazione dimostra come il sistema compositivo di Lucier possa diventare uno strumento da cui partire per nuove possibilità che mantengono identità sonore diverse rispetto al brano originario, pur mantenendo gli stessi processi.

Il problema dell'identità dell'opera emerge chiaramente quando ci si confronta con realizzazioni che utilizzano il sistema compositivo di Lucier come strumento per nuove creazioni. La realizzazione acusmatica ottenuta montando i risultati finali di tre stanze diverse - dove il processo di \textit{I am sitting in a room} è stato portato al punto di dissoluzione completa della voce in ciascun ambiente - solleva interrogativi fondamentali sui confini dell'opera. È mediante le differenze introdotte dall'interprete che si manifesta l'identità del sistema costruito dal compositore, e ciascuna realizzazione consiste in una manifestazione diversa dello stesso potenziale costruttivo. In questa prospettiva, il \textit{sistema Lucier} diventa uno strumento compositivo che mantiene la poetica originale pur generando identità sonore inedite.

Nella versione in tempo reale, l'ascoltatore assume un ruolo attivo nel processo. La sua presenza fisica influenza direttamente il risultato sonoro: variazioni di temperatura, umidità, movimenti nello spazio diventano parte del processo di trasformazione. L'interprete deve reagire a eventi imprevisti mantenendo il controllo del feedback per raggiungere l'obiettivo del brano.

Emerge inoltre nell'esecuzione del brano un'ulteriore domanda: quando termina l'opera? Il brano termina quando il suono della voce iniziale è ormai dileguato e le risonanze della stanza si sono manifestate chiaramente, e quando, nel passaggio da una \textit{generazione} alla successiva, il suono non sembra più modificarsi in alcun modo percettivamente sensibile. Questa decisione, affidata al giudizio dell'interprete sulla base di un criterio di \textit{efficienza musicale}, introduce un elemento di soggettività che contrasta con l'apparente automatismo del processo, rivelando come anche in un sistema deterministico l'elemento umano rimanga decisivo per la definizione dei confini formali dell'opera.