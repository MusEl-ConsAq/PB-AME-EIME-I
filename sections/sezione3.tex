% --- Contenuto LaTeX autogenerato da capitolo3.md (sezione 3) ---

\section{Feedback e trasformazione}
Il processo di trasformazione del materiale sonoro in \textit{I Am Sitting in a Room} si articola attraverso un meccanismo di feedback che coinvolge simultaneamente la voce parlata e le caratteristiche acustiche dello spazio. La voce iniziale, sottoposta al ciclo iterativo di registrazione-riproduzione, si fonde progressivamente con le risonanze naturali della stanza fino a dissolversi nelle frequenze caratteristiche dello spazio stesso.

Questa fusione avviene attraverso un processo di retroazione (feedback) che può essere formalizzato come una linea di ritardo con coefficiente di amplificazione. In termini matematici, il sistema si comporta come un filtro IIR (infinite impulse response) dove:

y(n) = A y(n-k) + x(n)

dove  y(n) rappresenta il segnale in uscita, che è pari alla somma del segnale attuale in ingresso x(n) e di quello fornito in uscita ad un dato tempo precedente y(n-k) che viene attenuato da un coefficiente A<1 (per non incorrere in accumulazione e saturazione). Si tratta di una linea di ritardo con feedback dove il ritardo è dato da k e il valore di feedback è A. Volendo riscrivere l'equazione rispetto alla realizzazione di Lucier, dove A rappresenta il valore di ingresso al secondo registratore e B il livello di uscita dal primo registratore:

y(n) = A y(n-k) + B x(n)

La durata del testo recitato assume un ruolo determinante nel processo. Come osservato da Lucier, stabilisce non solo la frequenza fondamentale teorica del processo (1/k), ma determina anche la \textit{finestra temporale} entro la quale le risonanze della stanza possono svilupparsi e decadere. Il rapporto tra coefficiente di feedback e durata del ciclo è fondamentale poiché definisce l'equilibrio tra persistenza del processo e controllo della saturazione: un valore di feedback troppo elevato causerebbe accumulazione fino alla saturazione, mentre un valore troppo basso farebbe estinguere rapidamente il processo di retroazione.

Il processo presenta caratteristiche \textit{autonome} che sollevano interrogativi fondamentali circa il controllo artistico. Una volta attivato, il sistema procede secondo le proprie leggi fisiche, ma il risultato finale dipende dalle specifiche caratteristiche della stanza, dalla qualità della voce, dalle condizioni ambientali, e da una serie di parametri che non possono essere completamente controllati o previsti. La distinzione tra processo  automatico e processo  autonomo è qui fondamentale: mentre un processo automatico segue istruzioni predeterminate senza possibilità di variazione, un processo autonomo sviluppa dinamiche proprie in interazione con l'ambiente e le condizioni di contorno.

Il processo è potremo definirlo stocastico: la lenta mutazione timbrica è sin dall'inizio predestinata a svolgersi in una sola direzione, ad uno stadio finale che in linea teorica è noto sin dall'inizio. Tuttavia, per ciascuna stanza e per ciascun testo adottati, il destino del processo non cambia ma il suo fenomeno rimane caratterizzato da variazioni che dipendono dalle variazioni nelle condizioni di contorno. Il compositore non controlla i dettagli dello svolgimento ma progetta le condizioni entro le quali il processo può svilupparsi autonomamente. L'intenzionalità si manifesta nella scelta dei parametri iniziali, nella selezione dello spazio, nella formulazione del testo iniziale. La casualità — intesa come insieme delle variazioni non controllabili che ogni esecuzione introduce — non rappresenta un elemento di disturbo ma costituisce il materiale stesso dell'opera, garantendo che ogni realizzazione sia un evento unico e irripetibile.
\subsection{Articolazione}
La tecnologia musicale sviluppata da Lucier nella sua opera, va oltre la tradizionale dicotomia mezzo/fine. I mezzi tecnici, da strumenti di riproduzione diventano strumenti di produzione e di pensiero, passando da una destinazione funzionale ad una destinazione creativa. La tecnologia non è più supporto neutro, ma diventa co-autore dell'opera insieme all'interprete e all'ambiente, generando fenomeni musicali che non sono né ottenibili né immaginabili con mezzi diversi.

L'architettura del processo si basa su un elemento fondamentale: la stanza che agisce come filtro posto tra ingresso e uscita del sistema. Delle frequenze presenti nello spettro della voce, risuoneranno in particolare quelle coincidenti con le proprie risonanze acustiche naturali. L'azione di filtro attuata dalla stanza deriva dalle interferenze costruttive (risonanze) e distruttive (antirisonanze) relative alle riflessioni sonore delle superfici presenti.

Il minimalismo di Lucier si manifesta in questa grande economia di mezzi abbinata a un pensiero estremamente chiaro circa la loro funzione compositiva. \textit{I am sitting in a room} fornisce un esempio molto fertile e caratteristico dove l'opera consiste con chiarezza nella messa in atto di un metodo di lavoro, mentre il fenomeno sonoro offerto all'ascolto consiste nella traccia udibile di questo mettere in atto.
\subsection{Voce e Spazio: reciprocità dinamica}
Nel corso del brano, la bipolarità caratteristica del materiale - la netta distinzione morfologica tra voce e stanza (suono e spazio) - viene dinamizzata attraverso il processo. Voce e spazio operano l'uno sull'altro: ciascuno dei due agisce sull'altro e lo modifica mentre viene dall'altro agito e modificato. Le risonanze della stanza vengono «articolate» attraverso la voce, che diventa così non solo materiale sonoro ma anche mezzo per sollecitare le caratteristiche acustiche dello spazio. Reciprocamente, le risonanze hanno per Lucier il compito di levigare o smussare l'articolazione della voce, trasformando la stanza in uno strumento che riduce progressivamente l'articolazione dinamica del parlato fino a produrre una \textit{texture} di frequenze lentamente variabile.

Questa interazione ridefinisce radicalmente il concetto di materialità musicale: voce e spazio, nella loro relazione dinamica, non sono affatto materiale neutro, perché è proprio lo svolgersi della loro interferenza nel tempo che si ascolta nell'intero svolgimento del brano. In quanto materiali particolari e non generici, voce e stanza sono mezzi: ciascuno dei due ha la funzione di mediare, a chi ascolta, qualcosa che riguarda l'altro. Il rapporto tra materiali sonori e processo realizzativo crea una bipolarità: l'ambiente è contemporaneamente mezzo di produzione e materiale compositivo, modificando dinamicamente voce e spazio.